\usepath[/home/adityam/Projects/presentations]

\enablemode[euler]
\environment env-fonts.tex

\startMPdefinitions
% Arrowhead Modifications for TAOCP. Copied from some webpage of Knuth. 
% These arrows are better in appearance than the default mp arrows.
ahangle := 65;
ahlength := 10 ;
vardef arrowhead expr p =
  save q, e, f, g; path q; pair e; pair f; pair g;
  e = point length p of p;
  q = gobble(p shifted -e cutafter makepath(pencircle scaled 2ahlength))
    cuttings;
  f = point 0 of (q rotated 0.5ahangle) shifted e;
  g = point length q of (reverse q rotated -0.5ahangle) shifted e;
  f .. {dir (angle direction length q of (q rotated 0.5ahangle) - 0.3ahangle)}e
    & e{dir (angle direction 0 of ((reverse q) rotated -0.5ahangle)+0.3ahangle)} .. g
enddef;

vardef arrowpath expr p =
  p
enddef;

def mfun_draw_arrow_path text t =
    if autoarrows :
        set_ahlength(t) ;
    fi
    draw arrowpath mfun_arrow_path t ;
    draw arrowhead mfun_arrow_path t ;
    endgroup ;
enddef ;

def mfun_draw_arrow_path_double text t =
    if autoarrows :
        set_ahlength(t) ;
    fi
    draw arrowpath (reverse arrowpath mfun_arrow_path) t ;
    draw arrowhead mfun_arrow_path t ;
    draw arrowhead reverse mfun_arrow_path t ;
    endgroup ;
enddef ;


  primarydef a -+ b =
    a -- (xpart b, ypart a) -- b
  enddef ;

  primarydef a +- b =
    a -- (xpart a, ypart b) -- b
  enddef ;

  input boxes;
  input mp-sketch;
  sketchypaths;
  sketch_amount := 1.3bp;
  pickup pencircle scaled 2bp;
  defaultdx := 10bp;
  defaultdy := 10bp;

\stopMPdefinitions


\starttext

\startMPpage[offset=1dk] % {{{ Feedback
  begingroup;
  sketchypaths;

  ux := 1cm;
  uy := 1cm;

  pair split_one, split_two;
  pair LTI_two;
  boxit.gain("$K$");
  boxit.LTI_one("$G(s)$");
  circleit.sum_one("\tfc $+$");

  split_one = origin;

  LTI_one.w - gain.e = gain.w - sum_one.e = (1.5ux,0);
  LTI_one.s - LTI_two = (0, uy);

  boxes_fixsize(sum_one,gain, LTI_one);
  boxes_fixpos (sum_one,gain, LTI_one);

  drawunboxed(sum_one,gain, LTI_one);

  pair Input, Output;

  Input  := split_one - (1.5ux,0);
  split_two := LTI_one.e + (ux,0);
  Output := split_two + (ux,0);

  for i = 1 upto 5 :
    sketch_amount := 1.3bp;
    draw bpath LTI_one cornered 10bp;
    draw bpath gain cornered 10bp;
    draw bpath sum_one;
    
    sketch_amount := 0.6bp;
    drawarrow Input -- sum_one.w;
    drawarrow sum_one.e -- lft gain.w ;
    drawarrow gain.e -- lft LTI_one.w;
    drawarrow LTI_one.e -- Output;
    drawarrow split_two +- LTI_two -+ bot sum_one.s ;
  endfor;

  label.ulft("\tfx $+$", sum_one.w + (2bp,2bp)); 
  label.lrt("\tfx $-$", sum_one.s + (2bp,0) ); 
  label.lft("$r(t)$", Input);
  label.rt("$y(t)$", Output);

  currentpicture := currentpicture scaled 1.5;

  endgroup;
\stopMPpage % }}}

\startMPpage[offset=1dk] % {{{ Example 1
  begingroup;
  sketchypaths;

  ux := 1cm;
  uy := 1cm;

  pair split_one, split_two;
  pair LTI_two;
  boxit.gain("$K$");
  boxit.LTI_one("$\dfrac{s+1}{s(s-1)(s+6)}$");
  circleit.sum_one("\tfc $+$");

  split_one = origin;

  LTI_one.w - gain.e = gain.w - sum_one.e = (1.5ux,0);
  LTI_one.s - LTI_two = (0, uy);

  boxes_fixsize(sum_one,gain, LTI_one);
  boxes_fixpos (sum_one,gain, LTI_one);

  drawunboxed(sum_one,gain, LTI_one);

  pair Input, Output;

  Input  := split_one - (1.5ux,0);
  split_two := LTI_one.e + (ux,0);
  Output := split_two + (ux,0);

  for i = 1 upto 5 :
    sketch_amount := 1.3bp;
    draw bpath LTI_one cornered 10bp;
    draw bpath gain cornered 10bp;
    draw bpath sum_one;
    
    sketch_amount := 0.6bp;
    drawarrow Input -- sum_one.w;
    drawarrow sum_one.e -- lft gain.w ;
    drawarrow gain.e -- lft LTI_one.w;
    drawarrow LTI_one.e -- Output;
    drawarrow split_two +- LTI_two -+ bot sum_one.s ;
  endfor;

  label.ulft("\tfx $+$", sum_one.w + (2bp,2bp)); 
  label.lrt("\tfx $-$", sum_one.s + (2bp,0) ); 
  label.lft("$r(t)$", Input);
  label.rt("$y(t)$", Output);

  currentpicture := currentpicture scaled 1.5;

  endgroup;
\stopMPpage % }}}

\startMPpage[offset=1dk] % {{{ Example 2
  begingroup;
  sketchypaths;

  ux := 1cm;
  uy := 1cm;

  pair split_one, split_two;
  pair LTI_two;
  boxit.gain("$A + \dfrac{B}{s}$");
  boxit.LTI_one("$\dfrac{1}{(s+1)(s+2)}$");
  circleit.sum_one("\tfc $+$");

  split_one = origin;

  LTI_one.w - gain.e = gain.w - sum_one.e = (1.5ux,0);
  LTI_one.s - LTI_two = (0, uy);

  boxes_fixsize(sum_one,gain, LTI_one);
  boxes_fixpos (sum_one,gain, LTI_one);

  drawunboxed(sum_one,gain, LTI_one);

  pair Input, Output;

  Input  := split_one - (1.5ux,0);
  split_two := LTI_one.e + (ux,0);
  Output := split_two + (ux,0);

  for i = 1 upto 5 :
    sketch_amount := 1.3bp;
    draw bpath LTI_one cornered 10bp;
    draw bpath gain cornered 10bp;
    draw bpath sum_one;
    
    sketch_amount := 0.6bp;
    drawarrow Input -- sum_one.w;
    drawarrow sum_one.e -- lft gain.w ;
    drawarrow gain.e -- lft LTI_one.w;
    drawarrow LTI_one.e -- Output;
    drawarrow split_two +- LTI_two -+ bot sum_one.s ;
  endfor;

  label.ulft("\tfx $+$", sum_one.w + (2bp,2bp)); 
  label.lrt("\tfx $-$", sum_one.s + (2bp,0) ); 
  label.lft("$r(t)$", Input);
  label.rt("$y(t)$", Output);

  currentpicture := currentpicture scaled 1.5;

  endgroup;
\stopMPpage % }}}

\startMPpage[offset=1dk] % {{{ Example 3
  begingroup;
  sketchypaths;

  ux := 1cm;
  uy := 1cm;

  pair split_one, split_two;
  pair LTI_two;
  boxit.gain("$K$");
  boxit.LTI_one("$\dfrac{6}{s^3 + 6s^2 + 11s + 6}$");
  circleit.sum_one("\tfc $+$");

  split_one = origin;

  LTI_one.w - gain.e = gain.w - sum_one.e = (1.5ux,0);
  LTI_one.s - LTI_two = (0, uy);

  boxes_fixsize(sum_one,gain, LTI_one);
  boxes_fixpos (sum_one,gain, LTI_one);

  drawunboxed(sum_one,gain, LTI_one);

  pair Input, Output;

  Input  := split_one - (1.5ux,0);
  split_two := LTI_one.e + (ux,0);
  Output := split_two + (ux,0);

  for i = 1 upto 5 :
    sketch_amount := 1.3bp;
    draw bpath LTI_one cornered 10bp;
    draw bpath gain cornered 10bp;
    draw bpath sum_one;
    
    sketch_amount := 0.6bp;
    drawarrow Input -- sum_one.w;
    drawarrow sum_one.e -- lft gain.w ;
    drawarrow gain.e -- lft LTI_one.w;
    drawarrow LTI_one.e -- Output;
    drawarrow split_two +- LTI_two -+ bot sum_one.s ;
  endfor;

  label.ulft("\tfx $+$", sum_one.w + (2bp,2bp)); 
  label.lrt("\tfx $-$", sum_one.s + (2bp,0) ); 
  label.lft("$r(t)$", Input);
  label.rt("$y(t)$", Output);

  currentpicture := currentpicture scaled 1.5;

  endgroup;
\stopMPpage % }}}


\stoptext

% vim: foldmethod=marker
